\section{Machine Learning als möglicher Levelgenerator}
\subsection{Hintergrund [wie ist Patrick drauf gekommen?]}
Im Gegensatz zu dem gegebenen Levelgenerator, wollten wir einen Ansatz wählen, der nicht auf prozedualer Generieriung beruht, sondern mit Machine Learning Techniken arbeitet. 

Der Generator soll anhand von bestehenden Leveln lernen und daraus neue generieren.


\subsection{Generative Adversial Networks (GAN)}
Generative Adverserial Networks, kurz GAN (zu deutsch "erzeugende generische Netzwerke"), stellen in der Informatik eine Gruppe von Algorithmen zu unüberwachtem Lernen dar. Sie bestehen aus zwei Neuronalen Netzwerken, eines erstellt Kandidaten (\textbf{Generator}), das zweite bewertet diese (\textbf{Diskriminator}).\\Der Generator lernt, Ergebnisse nach einer bestimmten Verteilung zu erzeugen. Der Diskriminator hingegen lernt, die Ergebnisse des Generators gegen die echte, vorgegeben Verteilung zu evaluieren (hier: konkret spielbare Level). Findet der Diskriminator keine Unterschiede mehr im direkten Vergleich der vorgegebenen Verteilung, so wird das Ziel erreicht.\\Neuronale Netzwerke kommen häufig zur Visualisierung verschiedener Gegenstände, zur Erstellung von 2D- bzw. 3D-Modellen oder zur Bildbearbeitung (astronomischer Bilder) zum Einsatz.\\Aus diesem Grund entschieden wir uns für GAN als geeigneten Kandidaten eines Level-Generators.
\subsection{...}