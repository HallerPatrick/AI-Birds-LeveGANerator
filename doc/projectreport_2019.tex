\documentclass[11pt]{article}
\usepackage[left=3cm,right=3cm,top=3cm,bottom=3cm]{geometry}
\usepackage[utf8]{inputenc}
\usepackage[T1]{fontenc}
\usepackage{amsmath}
\usepackage{amssymb}
\usepackage{hyperref}
\usepackage{graphicx}
\usepackage{xcolor,listings}

% Used to reference java classes
\newcommand{\class}[1]{{\color{green!50!black}\small\texttt{#1}}}
% Used to reference files or folders in the git/project directory
\newcommand{\source}[1]{{\color{blue!80!white}\texttt{\lstinline|#1|}}}

\begin{document}
	\begin{titlepage}
		\begin{center}
			\title{Project Report: BamBirds 2019\\
				{\small Creating  an Intelligent Game Playing Agent for Angry Birds}}
			\includegraphics[width=3cm,height=3cm]{img/logo.png}
			\author{Samet\\Akcabay\and
			Patrick\\Haller}
		\end{center}
	\end{titlepage}
\maketitle
\newpage
\tableofcontents
\newpage
\section{Aufgabenstellung des Smart Environment Projekts}
AngryBirds, ein in 2009 von Rovio Entertainment entwickeltes Casual-Puzzle-Videospiel, ist ideal als Maßstab für künstliche Intelligenz geeignet. Der Weg zum Erfolg setzt voraus, dass ein Verständnis der physikalischen Zusammenhänge auf das Spiel übertragen werden können und strategische Entscheidungen getroffen werden müssen. Ziel des Projekts war es, die Schwächen des BamBirds-Agenten zu verbessern und um Funktionen zu erweitern, um mit diesem an der Angry Birds AI Competition teilnehmen zu können und diesen zu gewinnen.\\Parallel zu diesem Wettkampf, lief die sogenannte AIBirds COG 2019 Level Generation Competition, in welchem das Ziel daraus bestand, mit einem selbsterstellten Programm der Wahl, automatisiert Level zu generieren, welche viel Spaß und Herausforderung anbieten sollen. Dieser Aufgabe haben wir uns gewidmet.
\section{GAN als möglicher Levelgenerator}
Generative Adverserial Networks, kurz GAN (zu deutsch "erzeugende generische Netzwerke"), stellen in der Informatik eine Gruppe von Algorithmen zu unüberwachtem Lernen dar. Sie bestehen aus zwei Neuronalen Netzwerken, eines erstellt Kandidaten (\textbf{Generator}), das zweite bewertet diese (\textbf{Diskriminator}).\\Der Generator lernt, Ergebnisse nach einer bestimmten Verteilung zu erzeugen. Der Diskriminator hingegen lernt, die Ergebnisse des Generators gegen die echte, vorgegeben Verteilung zu evaluieren (hier: konkret spielbare Level). Findet der Diskriminator keine Unterschiede mehr im direkten Vergleich der vorgegebenen Verteilung, so wird das Ziel erreicht.\\Neuronale Netzwerke kommen häufig zur Visualisierung verschiedener Gegenstände, zur Erstellung von 2D- bzw. 3D-Modellen oder zur Bildbearbeitung (astronomischer Bilder) zum Einsatz.\\Aus diesem Grund widmeten wir uns den GAN als Kandidat eines Level-Generators.
\section{GAN im Vergleich zu Alternativen}
...
\section{Allgemeine Vorgehensweise}
Im Folgenden wird die allgemeine Vorgehensweise beschrieben. Zur besseren Strukturierung, Dokumentierung und klaren Aufgabenverteilung, haben wir uns ein eigenes GitHub-Repository aufgesetzt, in welchem ebenfalls die Arbeitszeiten an den bestimmten Issues festgehalten wurden. Die wichtigsten Aufgaben werden in den nachfolgenden Unterkapiteln näher erläutert.
\subsection{Erkennen der  Konturen von Zentroiden}
...
\subsection{Parsen von und in XML bzw. JSON}
...
\subsection{Automatisierung der Abläufe mithilfe einer GUI und Powershell unter Windows}
Um zu vermeiden, dass alle erforderlichen Komponenten einzeln und umständlich gestartet werden müssen, haben wir es uns zur Aufgabe gemacht, ein Skript aufzusetzen, welches diesen Vorgang übernimmt. Dieses ist bisher auf Windows abgestimmt, kann aber bei Bedarf analog für Linux bzw. macOS mit eigenen Parametern erweitert werden.\\Das Skript startet zunächst ScienceBirds und skaliert diesen mithilfe von Window-Resizer (Tool zur nutzerbasierten Steuerung der Standardgröße eines Fensters, hier: ScienceBirds) auf einen bestimmten Wert, da der Agent sonst mit der Größe des ScienceBirds-Fensters nicht einverstanden ist. Im Anschluss wird der Agent und der Server automatisch in der Eingabeaufforderung gestartet. [...]
\subsection{Konvertierung von Pythons Pillow Koordinaten in ein kartesisches XML-Koordinatensystem}
\subsection{Verdeutlichen der Konturen der erzeugten RAW-Bilder zur besseren Erkennung vor Training}
Der Konturen-Erkenner hatte in unserem Durchlauf Probleme, die Umrisse der erzeugten Zentroide zu erkennen. Aufgrund dessen mussten wir die Konturen, welche erzeugt werden, vor dem Beginn des Trainings verstärken, damit diese eindeutig von unserem Erkenner gefasst werden können. [...]
\subsection{Aufsetzen eines Re-Evaluierungssystems}
\section{Ergebnisse}
Aus unseren Ergebnissen wird deutlich, dass GAN weniger zum Generieren von Level geeignet ist, da die erzeugten Level nicht sinnvoll sind. Das heißt, dass häufig obsolete Platforms vorzufinden sind, welche in ihrem Aufbau und ihrer Struktur kein spielbares Level darstellen. Zudem erfolgt die Platzierung der Schweine und Hindernisse ohne jeglichen Wert auf Faktoren wie Spaß und Herausforderung zu legen, weswegen die generierten Level wenig für den Spieler bieten zu haben. 
\section{Fazit}

\section{Ausblick}
[Verbesserungsvorschläge][Wie kann man GAN passend für Level-Generierung gestalten?] [Mögliche Alternativen?]
\end{document}
