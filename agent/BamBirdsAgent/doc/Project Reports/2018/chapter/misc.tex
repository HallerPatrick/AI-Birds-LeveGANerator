\section{Miscellaneous}\label{ch:misc}

\subsection{Including a logger for debugging}\label{subsec:logger}
The agent gives multiple lines of output during its gameplay. Some of it is only necessary during development, but not during the competition. Text outputs also require (a quite small amount of) cpu time. In order to save this time and declutter the agent's output during the competition, we implemented a \texttt{CustomLogger.class} that instanciates a \texttt{java.util.logging.Logger} and a custom message formatter. It listens to the constant \texttt{DEBUG\_ENABLED} inside the \texttt{Constants.class}. If this value is set to true, every message of the agent will be printed. If it is set to false, only messages that are marked as \textit{warning} or \textit{severe} will be output.

In order to keep this ability, no more \texttt{System.out.println()} calls should be used. Instead, one should use \texttt{CustomLogger.[info/warning/severe]([message string])} in order to output messages. One can also, at any time, call \texttt{CustomLogger.setLevel(["info"/"warning"/"severe"])} in order to change the log level. Since the logger is implemented as a singleton, every class that instantiates a \texttt{CustomLogger} object will work on the same instance.\\
On starting the agent, in \texttt{BamBird.class} the function \texttt{CustomLogger.saveLogsToFile()} is called. This effects that every message will be output, but also saved in \texttt{logfile.log}. 

A typical output of the logger would be

\ \\
\begin{centering}
\texttt{[2018-08-10 15:12:30] [INFO] meta.Meta | Choosing new level ... } \\
\end{centering}
\ \\
It shows a date- and timestamp, \texttt{[INFO]} marks the log level, \texttt{meta} is the package and \texttt{Meta} the class from which the logger is called, followed by the actual message "\texttt{Choosing new level}".

\subsection{Documentation tool}\label{subsec:code documentation}
For a clear dokumentation there is the posibility to use the documentation generator 'Doxygen'. Hence it is necessary to install Doxygen and GraphViz. To build the documentation you have to open your terminal/bash in the documentation directory and run 'doxygen configfile'.
The configfile is a default by our project group but you can also tweak the settings by yourself. Therefore you have to use 'Doxywizard'(Open the terminal/bash and run 'doxywizard'). The graphical user interface is easy to handle and settings can be changed\footnote{https://www2.informatik.hu-berlin.de/swt/projekt98/werkzeuge/doxygen/Doxygen.html}.
The documentation in the programming code has to keep some rules: \\

Writing '/**' and enter creates automatically the main functions like: \\
@brief - short brief of the code \\
@param - description of the parameter \\
@return - description of the return value \\
@class - description of the class \\

Further information on how to use doxygen can be found in the following document\footnote{http://www.vislab.de/cgbuch/intros/doxygen.pdf}.